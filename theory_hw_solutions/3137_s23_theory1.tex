\documentclass{article}
\title{COMS W3137\\ Solutions for Theory Homework 1}
\begin{document}

\setlength{\parindent}{0in}
\maketitle


\section*{Problem 1}
We want to show $4 n^2 + 2n - 1 = \Theta(n^2)$. We must find positive constants $c_1$, $c_2$, and $n_0$, such that 
$c_1 n^2 \leq 4 n^2 +2n - 1 \leq c_2 n^2$ for all $n \geq n_0$. Dividing by $n^2$ yields
$$c_1 \leq 4 + \frac{2}{n} - \frac{1}{n^2} \leq c_2.$$

$ \frac{2}{n} - \frac{1}{n^2}$ approaches 0 as $n$ increases, so $\lim\limits_{n \rightarrow \infty} 4 + \frac{2}{n} - \frac{1}{n^2} = 4$. So we can set (for example) $c_2 = 5$ to satisfy the right side of the inequality. For the left side, we can set (again, for example) $c_1 = 1$ and $n_0 = 1$. To verify: 
$$4 + \frac{2}{n_0} - \frac{1}{n_0^2} = 5.$$
Other proofs (and other constants) are possible. 

\section*{Problem 2}

Given a polynomial $p_d(n) = \sum_{i=0}^{d} a_i n^i$
 and some integer $k  \geq d$, We need to show that 

$$\sum_{i=0}^{d} a_i n^i \leq c n^k  \textit{ for all } n\geq n_0.$$

Because $k \geq d$, we know $n^k \geq n^i$ for each term $n^i$, therefore

$$\sum_{i=0}^{d} a_i n^i \leq  \sum_{i=0}^{d} a_i n^k$$\\ factor out $n^k$:
$$ \sum_{i=0}^d a_i n^i \leq  n^k \sum_{i=0}^d a_i.$$

So we can set $c =   \sum\limits_{i=0}^d a_i$ and $n_0=1$ and satisfy the definition for big-O.

Other proofs are possible.

\section*{Problem 3}
$$2/N <128 < \log N < \sqrt{N} < 3N <N \log N < N^2 < 7 N^3 < 2^N = 2^{N+1} < 4^n < N!$$

Note that $lim_{n\rightarrow \infty} \frac{2^{n+1}}{2^n} = 2$ and therefore $2^n  = \Theta(2^{n+1})$, but $lim_{n\rightarrow \infty} \frac{4^{n}}{2^n} = 2^n$, so $2^n = o(4^n)$.

\section*{Problem 4}

\begin{itemize}
\item[a)] 
$$D = 2^{2^{(N-1)}}$$

\item[b)] Solve the equation from a) for $N$.

$$D = 2^{2^{(N-1)}}$$
$$log_2 D = 2^{(N-1)} $$
$$log_2 \log_2 D = N-1$$
$$N = \log_2 \log_2 D  + 1$$
 
Therefore $N = \Theta{(\log \log D)}$.
\end{itemize}

\section*{Problem 5}

\begin{itemize}
  \item[a)] The inner loop runs a constant number of times, so only the outer loop depends on $n$. The runtime is $O(n)$.
  \item[b)] In the first iteration of the outer loop, the inner loop will not run. In the second iteration of the outer loop, the inner loop performs $O(n^2)$ steps, but now $k = n^2$, so the outer loop will not enter into another iteration. The the total runtime is $O(n^2)$. 
  \item[c)] The function will call itself recursively $log_c n$ times before the base case is reached. The runtime is $O(\log n)$
\end{itemize}

\end{document}
